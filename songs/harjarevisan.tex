\begin{song}

\songtitle{Härjarevisan}
\firstline{Liksom våra fäder vikingarna i Norden}
\begin{songmeta}
Melodi: Gärdebylåten
\end{songmeta}

\begin{songtext}
Liksom våra fäder vikingarna i Norden
drar vi riket runt och super oss under borden.
Brännevin har blitt ett elixir för kropp såväl som själ.
Känner du dig liten och ynklig på jorden
växer du med supen och blir stor uti orden,
slår dig för ditt håriga bröst
och blir en man från hår till häl.

Ja, nu ska vi ut och härja,
supa och slåss och svärja,
bränna röda stugor,
slå små barn och säga fula ord.
Med blod ska vi stäppen färga,
nu änteligen lär ja'
kunna dra nån verklig nytta
av min Hermodskurs i mord.

Hurra, nu ska man äntligen få röra på benen.
Hela stammen jublar och det spritter i grenen.
Tänk att än en gång få spränga fram på Brunte i galopp.
Din doft, o kära Brunte, är trots brist i hygienen
för en vild mongol minst lika ljuv som syrenen.
Tänk att på din rygg få rida runt i stan och spela topp!

Ja, nu ska vi ut och härja\ldots


Ja, mordbränder är klämmiga, ta fram fotogenen.
Eftersläckningen tillhör just de fenomenen
inom brandmansyrket, som det är nån verklig nytta med.
Jag målar för mitt inre upp den härliga scenen:
Blodrött mitt i brandgult, ej ens prins Eugen en
lika mustig vy kan måla, ens om han målade med sked.

Ja, nu ska vi ut och härja\ldots
\end{songtext}

\begin{songnotes}
Ur Lundaspexet \textquotedblleft{}Djingis Khan\textquotedblright{}, 1954. \\
Endast andra och tredje versen härrör ur spexet ifråga. Den förstas ursprung är okänt.
\end{songnotes}

\vfill
\begin{figure}[h!]
  \centering
  \includegraphics[scale=0.3]{images/de_sjutton_suparnas_intagande}
\end{figure}
\vfill

\end{song}

\begin{song}

\songtitle{Den blomstertid nu kommer}

\begin{songmeta}
Melodi: Den blomstertid nu kommer
Text: Israel Kolmodin
Musik: Trad.
\end{songmeta}

\begin{songtext}
Den blomstertid nu kommer
med lust och fägring stor.
Du nalkas, ljuva sommar,
då gräs och gröda gror.
Med blid och livlig värma
till allt som varit dött,
sig solens strålar närma,
och allt blir återfött.

De fagra blomsterängar
och åkerns ädla säd,
de rika örtesängar
och lundens
gröna träd,
de skola oss påminna
Guds godhets rikedom,
att vi den nåd besinna
som räcker året om.
\newpage
Man hörer fåglar sjunga
med mångahanda ljud,
skall icke då vår tunga
lovsäga Herren Gud?
Min själ, upphöj Guds ära,
stäm upp din glädjesång
till den som vill oss nära
och fröjda på en gång!
\end{songtext}

\begin{songnotes}
En av de mest kända svenska psalmerna, vars text trycktes redan i Den svenska psalmboken 1695. Den har därefter omarbetats ett flertal gånger, senast 1979.
Det finns tre verser till, men de sjungs sällan, så de har inte tagits med här.
Sv. psalm nr 199
\end{songnotes}

\end{song}

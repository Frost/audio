\begin{song}

\songtitle{Om Haga}
\firstline{Fjäriln vingad syns på Haga}

\begin{songmeta}
Alternativ titel: Fjäriln vingad
Alternativ titel 2: Fjäriln vingad syns på Haga
Alternativ titel 3: Fredmans sång no. 64
Melodi: Om Haga
Text och Musik: Carl Michael Bellman
\end{songmeta}

\begin{songtext}
Fjäriln vingad syns på Haga,
mellan dimmors frost och dun
sig sitt gröna skjul tillaga
och i blomman sin paulun.
Minsta kräk i kärr och syra,
nyss av solens värma väckt,
till en ny högtidlig yra
eldas vid sefirens fläkt.

Haga i ditt sköte röjes
gräsets brodd och gula plan.
Stolt i dina rännlar höjes
gungande den vita svan.
Längst ur skogens glesa kamrar
höras täta återskall
än från den graniten hamrar,
än från yx i björk och tall.
\newpage
Se Brunnsvikens små najader
höja sina gyllne horn,
och de frusande kaskader
sprutas över Solna torn.
Under skygd av välvda stammar
på den väg man städad ser,
fålen yvs och hjulen dammar,
bonden milt åt Haga ler.

Vad gudomlig lust att röna
inom en så ljuvlig park,
då man hälsad av sin sköna
ögnas av en mild monark!
Varje blick hans öga skickar,
lockar tacksamhetens tår;
rörd och tjust av dessa blickar,
själv den trumpne glättig går.
\end{songtext}

\begin{songnotes}
Dediceras till herr capitainen Kjerstein
\end{songnotes}

\end{song}

\begin{song}

\songtitle{Kungssången}
\firstline{Ur svenska hjärtans djup en gång}

\begin{songmeta}
Alternativ titel: Ur svenska hjärtans djup en gång
Melodi: Kungssången
Text: Carl Wilhelm August Strandberg
Musik: Otto Lindblad
\end{songmeta}

\begin{songtext}
Ur svenska hjärtans djup en gång
en samfälld och en enkel sång,
som går till kungen fram!
Var honom trofast och hans ätt,
gör kronan på hans hjässa lätt,
och all din tro till honom sätt,
du folk av frejdad stam!

Du himlens Herre, med oss var,
som förr du med oss varit har,
och liva på vår strand
det gamla lynnets art igen
hos sveakungen och hans män.
Och låt din ande vila än
utöver nordanland!
\end{songtext}

\newpage
\begin{songnotes}
Här har endast första och sista versen tagits med. Tre verser till finns, men
sjungs mer sällan. \\
Kungssången skrevs till och uruppfördes vid en studentfest i Lund den 5 december
1844 i anslutning till Lunds \\ universitets firande av Oscar I:s tronbestigning.
Tidigare hade det varit brukligt att byta kung och kungssång på samma gång,
varför auditoriet, som tog för givet att detta var den nya kungssången, reste
sig och sjöng med. Fastän det inte är klarlagt att detta var vad
upphovsmännen hade avsett, har sången därefter fått gälla som kungssång. \\
Till skillnad från exempelvis den brittiska ``God Save the Queen'' har den
svenska kungssången inte status av nationalsång. Förutom Sverige har även
Danmark, Norge och Thailand särskilda kungssånger.
\end{songnotes}

\end{song}

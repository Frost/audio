\begin{song}

\songtitle{Luciasången}
\firstline{Natten går tunga fjät}

\begin{songmeta}
Alternativ titel: Sankta Lucia
Alternativ titel 2: Natten går tunga fjät
Melodi: Luciasången
Text: Arvid Rosén
Musik: Trad.
\end{songmeta}

\begin{songtext}
Natten går tunga fjät
runt går och stuva,
kring jord som sol'n förlät
skuggorna ruva.

||: Då i vårt mörka hus
stiger med tända ljus
Sankta Lucia,
Sankta Lucia. :||

Natten var stor och stum.
Nu hör det svingar
i alla tysta rum
sus som av vingar.

||: Se, på vår tröskel står,
vitklädd, med ljus i hår,
Sankta Lucia,
Sankta Lucia. :||

\textquotedblleft{}Mörkret skall flykta snart
ur jordens dalar.\textquotedblright{}
Så hon ett underbart
ord till oss talar.

||: Dagen skall åter ny
stiga ur rosig sky,
Sankta Lucia,
Sankta Lucia. :||
\end{songtext}

\begin{songnotes}
Arvid Rosén ville troligtvis att texten skulle låta som en ålderdomlig folkvisa, och använde sig \\ således av ord som stuva och förgät, ord som var svårbegripliga redan 1928, då texten skrevs. \\\textquotedblleft{}Suset\textquotedblright{} i andra versen är lånat från den andra textversionen.
\end{songnotes}

\end{song}

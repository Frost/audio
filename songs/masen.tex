\begin{song}

\songtitle{Måsen}
\firstline{Det satt en mås på en klyvarbom}
\begin{songmeta}
Melodi: När månen vandrar på fästet blå (Trad.)
\end{songmeta}

\begin{songtext}
Det satt en mås på en klyvarbom
och tom i krävan var kräket.
Och tungan lådde vid skeppar'ns gom
där skutan låg uti bleket.
\textquotedblleft{}Jag vill ha sill!\textquotedblright{} hördes måsen rope,
men skeppar'n svarte: \textquotedblleft{}Jag vill ha O.P.
om blott jag får,
om blott jag får\textquotedblright{}.

Nu lyfter måsen från klyvarbom
och vinden spelar i tågen.
O.P.:n svalkat har skeppar'ns gom.
Jag önskar blott att jag såg 'en.
Han är så lycklig den gamle saten.
Han hissar storseglet den krabaten.
Till sjöss han far\ldots
\ldots{}och halvan tar.

Den mås som satt på en klyvarbom,
den är nu död och begraven,
och skeppar'n som drack en flaska rom,
han har nu drunknat i haven.
Så kan det gå om man fått för mycke',
om man för brännvin har fattat tycke.
Vi som har kvar,
vi resten tar!
\end{songtext}

\begin{songnotes}
O.P. Anderson Aquavit är uppkallad efter spritfabrikören Olof Peter Anderson
(1797-1876). Drycken lanserades av hans son, Carl August, vid
Göteborgsutställningen 1891. Skåne Akvavit innehåller samma sorts kryddor, men
i bara ungefär halva mängden.
\end{songnotes}

\end{song}

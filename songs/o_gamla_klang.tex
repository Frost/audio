\begin{song}

\songtitle{O, gamla klang- och jubeltid!}
\begin{songmeta}
Melodi: O alte Burschenherrlichkeit (Eugen Höfling)
Text: August Lindh
\end{songmeta}

\begin{songtext}
O, gamla klang och jubeltid,
ditt minne skall förbliva,
och än åt livets bistra strid
ett rosigt skimmer giva.
Snart tystnar allt vårt yra skämt,
vår sång blir stum, vårt glam förstämt;
O, jerum, jerum, jerum,
O, quæ mutatio rerum!

Var äro de som kunde allt,
blott ej sin ära svika,
som voro män av äkta halt
och världens herrar lika?
De drogo bort från vin och sång
till vardagslivets tråk och tvång;
O, jerum, jerum, jerum,
O, quæ mutatio rerum!

Den ene vetenskap och vett
in i scholares mängder,
Den andre i sitt anlets svett
på paragrafer vränger,
en plåstrar själen som är skral,
en lappar hop dess trasiga fodral;
O, jerum, jerum, jerum,
O, quæ mutatio rerum!

Men hjärtat i en sann student
kan ingen tid förfrysa.
Den glädjeeld, som där han tänt,
hans hela liv skall lysa.
Det gamla skalet brustit har,
men kärnan* finnes frisk dock kvar,
och vad han än må mista,
den skall dock aldrig brista.

Så sluten, bröder, fast vår krets
till glädjens värn och ära!
Trots allt vi tryggt och väl tillfreds
vår vänskap trohet svära.
Lyft bägarn högt och klinga, vän!
De gamla gudar leva än
bland skålar och pokaler,
bland skålar och pokaler!
\end{songtext}

\begin{songnotes}
* Endast vid \textquotedblleft{}kärnan\textquotedblright{} dunkas näven i bordet en gång. Ingen annanstans.\\
Den sista versen sjungs stående, och efter det att sista versen sjungits sätter sig ingen åter till bords.\\
Meningen med denna sång är att man ska ställa sig upp där man känner sig kallad.\\
Jerum = Tysk omskrivning av \textquotedblleft{}Herr Jesus\textquotedblright{} (jfr. Jösses)\\
O, quæ mutatio rerum! = O, vilken förändring i allt!\\
Quæ uttalas egentligen \textquotedblleft{}kvä\textquotedblright{}, därav ligaturerna. \\
Det finns alternativa tredjeverser som sjungs vid olika lärosäten.
\end{songnotes}

\end{song}


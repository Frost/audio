\begin{song}

\songtitle{Jag har aldrig var't på snusen}
\begin{songmeta}
Melodi: O, hur saligt att få vandra
Text: Joël Blomqvist, Per Ollén
\end{songmeta}

\begin{songtext}
Jag har aldrig var't på snusen,
aldrig rökat en cigarr, halleluja!
Mina dygder äro tusen,
inga syndiga laster jag har.
Jag har aldrig sett nå't naket,
inte ens ett litet nyfött barn.
Mina blickar går mot taket;
därmed undgår jag frestarens garn.

||: Halleluja! :|| (ett otal gånger)

Bacchus spelar på gitarren.
Satan spelar på sitt handklaver.
Alla djävlar dansar tango.
Säg, vad kan man väl önska sig mer?
Jo, att alla bäckar vore brännvin,
Riddarfjärden* full av bayerskt öl,
konjak i varenda rännsten
och punsch i varendaste pöl.

||: Och mera öl! :||
\newpage
\end{songtext}

\begin{songnotes}
* Människor från andra städer sjunger ofta varianter på just detta ord i texten,
såsom \textquotedblleft{}Näckrosdammen\textquotedblright{}, \textquotedblleft{}hela Svartån\textquotedblright{} m.m. På KTH hörs ofta \\
\textquotedblleft{}Gasqueparksdammen\textquotedblright{}, vilket är namnet på fontänen utanför kårhuset Nymble.
Det rekommenderas då att du klämmer i ordentligt, så att de stackars oupplysta \\
själarna lär sig hur texten egentligen skall vara.

Att vomera öl är tämligen ouppfostrat.
\end{songnotes}

\end{song}

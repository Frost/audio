\begin{song}
\songtitle{Ingenjörssektionens sång}
\alttitle{Ingenjörssektionens sång}
\firstline{Om du vill gå på KTH}

\begin{songmeta}
Melodi: Fredmans sång nr. 21 (Carl Michael Bellman)
Sektion: Ingenjörssektionen
\end{songmeta}

\begin{songtext}
Om du vill gå på KTH
men komma ut väl fort ändå,
du slipper alla gråa hår
- examen på tre år.
Betänk ditt linjeval min vän,
så slipper du bli kvar här sen.
När vi på IS har gått ut
ser resten inget slut.

Tycker du att civ-ing är för lång,
eller Handels kanske är för vrång?
Gå då ett år och två år och tre år på Ing;
du bliver nöjdare.

Så tentar vi så småningom
i samma ämne gång på gång.
Examen ska vi en gång nå,
för vi på IS gå.

Tycker du\ldots
\end{songtext}

\begin{songnotes}
Det finns egentligen fyra ingenjörssektioner (Campus, Telge, Haninge och Kista),
men för att göra en lång historia kort var de ursprungligen en och samma sektion
och har därför såväl gemensam sektionsvisa som \\ gemensam logotyp.
\end{songnotes}
\end{song}
